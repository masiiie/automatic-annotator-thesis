\chapter*{Recomendaciones}\label{chapter:recomendaciones}

De manera general se considera que la investigaci\'on ha sido fruct\'ifera pero igualmente han  quedado pendientes varios aspectos importantes que se considera deben tenerse en cuenta para la continuidad del proyecto:
\begin{enumerate}
\item Construir un corpus que: 
\begin{enumerate}
\item Est\'e rigurosamente anotado por los ling\"uistas\footnote{Puede ser de igual tama\~no al del presente proyecto, pero l\'ogicamente si es mayor los resultados ser\'an mejores.}.
\item Contemple muestras de varios informantes. Este aspecto es crucial y se ve claramente en el Volumen II de "{Aspectos de la entonaci\'on}" que las grabaciones de un mismo entonema var\'ian seg\'un el hablante.
\item Atienda con mayor precausi\'on las t\'ecnicas de normalizaci\'on explicadas por River\'on, en cuanto a trabajar con medidas relativas en lugar de absolutas. 
\end{enumerate} 
\item Crear un modelo en el cual la vectorizaci\'on de los audios contemple tambi\'en otros rasgos estudiados por Garc\'ia River\'on, como el tiempo voc\'alico relativo y m\'aximo, la intensidad m\'axima, la velocidad del tono fundamental y el diapas\'on.
\item Experimentar la vectorizaci\'on de la muestra con otras wavelets como la de \emph{Haar}\cite[secci\'on 3.2.5]{addison2017illustrated}.
\item Construir un conjunto \textit{Test} lo m\'as representativo posible.
\end{enumerate}