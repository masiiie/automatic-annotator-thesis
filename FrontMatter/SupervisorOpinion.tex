%\chapter*{Opini\'on del tutor}
\begin{opinion}
La ling\"u\'istica computacional es una rama multidisciplinaria que procesa grandes cantidades de textos escritos u orales. En particular, el an\'alisis autom\'atico de textos orales es un reto y se ha centrado en la transcripci\'on ortogr\'afica. El estudio computacional de la entonaci\'on es menos tratado y no abundan los trabajos sobre este t\'opico en habla hispana, idioma con particularidades en este aspecto. 


La tesis presentada por Masiel Villalba Carmenate propone un algoritmo basado en aprendizaje autom\'atico para la clasificaci\'on de segmentos de habla del espa\~nol de Cuba en la tipolog\'ia de entonemas para esta variante del espa\~nol propuesta por la enton\'ologa cubana Dra.C. Raquel Mar\'ia Garc\'ia River\'on. Los entonemas son unidades presentes en la cadena hablada que cumplen funciones sintagm\'aticas y pragm\'aticas, y que interact\'uan con otros niveles ling\"u\'isticos (como la gram\'atica) para construir un significado ling\"u\'istico complejo. Esto hace muy complicado su estudio.


Uno de los inconvenientes afrontados durante la investigaci\'on fue la ausencia de un corpus anotado con la tipolog\'ia de entonemas usada. Por ello, uno de los primeros pasos fue la construcci\'on de dicho corpus. Adem\'as, Masiel sistematiz\'o algunos de los modelos computacionales utilizados para el estudio de la entonaci\'on hisp\'anica, ninguno compatible con el de la Dra.C. Garc\'ia River\'on. Como resultado fundamental se entren\'o y valid\'o un modelo de aprendizaje autom\'atico para clasificar segmentos del habla en la tipolog\'ia de entonemas usada.


Durante el desarrollo del trabajo, Masiel tuvo que estudiar la materia referida, que no est\'a incluida en el curr\'iculo de la carrera y trabaj\'o con ling\"uistas y otros especialistas de la computaci\'on, mostrando disciplina, entrega y rigor. Adem\'as, demostr\'o habilidades para el trabajo con la bibliograf\'ia y  creatividad para proponer soluciones a problemas de implementaci\'on, entre otras competencias de programaci\'on en el lenguaje Python y sus diversos frameworks. Es de destacar el esfuerzo realizado en la construcci\'on del corpus, as\'i como la adaptaci\'on creativa de algunas herramientas para realizar el an\'alisis. Masiel se ha superado y ha sorteado inconvenientes personales e investigativos para lograr terminar esta tesis, donde considero cumpli\'o el objetivo propuesto.



Por tanto, considero que a esta tesis de la estudiante Masiel Villalba Carmenate debe otorg\'arsele la m\'axima calificaci\'on (5 puntos, Excelente), y estoy seguro que en el futuro Masiel se desempe\~nar\'a como una excelente profesional de la Ciencia de la Computaci\'on.


\vfill
\begin{flushleft}
\underline{\hspace{140pt}}\hfill \underline{\hspace{140pt}}\\
MSc. Damian Vald\'es Santiago \hfill Dra.C. Raquel Mar\'ia River\'on \hspace*{7pt}
\end{flushleft}
\end{opinion}