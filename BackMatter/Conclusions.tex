\chapter*{Conclusiones}\label{chapter:conclusiones}
Se observ\'o finalmente, que el aprendizaje autom\'atico arroj\'o buenos resultados para el conjunto de entrenamiento y no muy alentadores para el conjunto de prueba tomado. Esto da la medida de que si se cuenta con un corpus con suficientes ejemplos por cada categor\'ia se pueden mejorar las puntuaciones, y que adem\'as, debe construirse un conjunto de prueba m\'as extenso y representativo. Desde una mirada optimista, se dice que se pueden confiar en las predicciones para los entonemas VE-1c, E-3, E-3b y VE-7a, para los cuales el modelo reporta mayor presici\'on.


Aunque cabe seguir depurando el sistema, se demostr\'o en la pr\'actica una gran coincidencia con los resultados de Garc\'ia River\'on, cuando advierte que los rasgos que mejor discriminan entre entonemas son la forma, el registro y la figura, puesto que considerando solo estos par\'ametros se lograron indentificar correctamente varias categor\'ias.

\begin{comment}
Malo que bueno, se ha construido el andamiaje para perfeccionar la solucion del problema y se ha capacitado a especialistas de la computacion.

exploratorio

- sirve para detectar los entonemas tal y tal

Lo de las curvas de validacion y eso da bien por el data augmentation pero no generaliza bien
una conclusion q si esta buena es decir q estos features no sirven para la clasificacion solo para ayudar a detectar el entonema 1c
\end{comment}